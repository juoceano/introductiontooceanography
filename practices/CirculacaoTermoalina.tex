%-----------------------------------------------------------------------------%
%	Packages & Other Configurations
%-----------------------------------------------------------------------------%
\RequirePackage{fix-cm}  % Fix Font shape `OT1/cmr/m/n' size substitution.
\documentclass[a4paper,10pt]{article}
\usepackage[top=0.5in, bottom=0.6in, left=1in, right=0.9in]{geometry}
\usepackage[utf8]{inputenc} %add acents
\usepackage{setspace} % command \doublespacing etc...
\usepackage{lineno} % number lines
\usepackage{epsf,epsfig} % includegraphics [pdf, png etc]
\usepackage{amsmath} %adicionei esse pacote pra vc poder usar o draft%
\usepackage{textcomp} %símbolos de texto
\usepackage{natbib} % bibtex - adicionar referencia
% \usepackage{url} % for bibtex - configuracoes de urls
\usepackage{tabularx} % for tables
\usepackage[hidelinks]{hyperref}  % Add URL links.
% \usepackage[bookmarks=false,colorlinks=true,urlcolor={green},linkcolor={green},pdfstartview={XYZ null null 1.22}]{hyperref} %all references

%-----------------------------------------------------------------------------%
%	Adicionar a Watermark
%-----------------------------------------------------------------------------%
\usepackage{draftwatermark}
\SetWatermarkAngle{45}
\SetWatermarkLightness{0.9}
\SetWatermarkFontSize{5cm}
\SetWatermarkScale{0.3}
\SetWatermarkText{Exercício - Oceanografia}

%-----------------------------------------------------------------------------%
%	Informações sobre o PDF
%-----------------------------------------------------------------------------%

\pdfinfo{%
  /Title    (GEO232 - Exercício)
  /Author   (Ju Leonel)
  /Creator  (Ju Leonel)
  /Producer (Ju Leonel)
  /Subject  (Intro oceanografias)
  /Keywords (Intro oceanografia, exercício circulação termoalina)}

%-----------------------------------------------------------------------------%
%	Documento
%-----------------------------------------------------------------------------%
\title{GEO232 - Introdução à Oceanografia - IGEO-UFBA}
\author{\vspace{-10ex}}
\date{\vspace{-10ex}}

\begin{document}

  \maketitle
  %\doublespacing
  \onehalfspace

  \begin{tabular*} {0.9\textwidth}{@{\extracolsep{\fill} } l l}
    \hline
    Professora: Juliana Leonel & Atendimento: Sextas-feiras \\
    E-mail: \href{mailto:jleonel@ufba.br}{jleonel@ufba.br} & Horário atendimento: 13:00-14:00 \\
    Aulas: Quintas e Sextas-Feiras & Local atendimento: IGEO - Sala 10 - 2\textsuperscript{o} andar\\
    Horário- Aulas: 7:30 - 10:40 & Homepage: \url{http://juoceano.github.io/chemicaloceanography2/}\\
    \hline
  \end{tabular*}

  \vspace{3ex}

    \section{Exercício: Circulação Termoalina}

    \noindent

    Um tanque de dimensões 10 m $\times$ 10 m $\times$ 10 m esta parcialmente cheio de água. No canto superior esquerdo é adicionado água com temperatura menor e salinidade maior que a água contida no tanque (chamaremos essa água de ÁGUA 1).

    \begin{itemize}
     \item [a)] O que acontece assim que a ÁGUA 1 é adicionado ao tanque?
     \item [b)] Ao deslocar-se a ÁGUA 1 ``empurra'' as demais parcelas de água presentes no tanque. Descreva e represente graficamente como será o movimento no tanque?
     \item [c)] Esse movimento é similar a alguns que ocorrem nos oceanos. Cite um desses movimentos.
    \end{itemize}
    
    * Ignorem qualquer desvio da partícula em função da rotação da Terra.

%\clearpage %termina o texto e tudo que estiver flutuante se encaixa ali
%\newpage % começa uma nova página
%\bibliographystyle{chicago} % estilo que vai sair a bibliografia (posso usar outros)
%\bibliography{references_ju} %entre chaves vai o nome do arquivo de referencias


\end{document}
