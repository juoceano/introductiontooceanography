\RequirePackage{fix-cm}  % Fix Font shape `OT1/cmr/m/n' size substitution.
%-----------------------------------------------------------------------------%
%	Packages & Other Configurations
%-----------------------------------------------------------------------------%
\documentclass[a4paper,10pt]{article}
\usepackage[top=0.5in, bottom=0.6in, left=1in, right=0.9in]{geometry}

\usepackage[utf8]{inputenc} %add acents
\usepackage{setspace} % command \doublespacing etc...
\usepackage{lineno} % number lines
\usepackage{epsf,epsfig} % includegraphics [pdf, png etc]
\usepackage{amsmath} %adicionei esse pacote pra vc poder usar o draft%
\usepackage{textcomp} %símbolos de texto
\usepackage{natbib} % bibtex - adicionar referencia
% \usepackage{url} % for bibtex - configuracoes de urls
\usepackage{tabularx} % for tables
\usepackage[hidelinks]{hyperref}  % Add URL links.
% \usepackage[bookmarks=false,colorlinks=true,urlcolor={green},linkcolor={green},pdfstartview={XYZ null null 1.22}]{hyperref} %all references


%-----------------------------------------------------------------------------%
%	Adicionar a Watermark
%-----------------------------------------------------------------------------%
\usepackage{draftwatermark}
\SetWatermarkAngle{45}
\SetWatermarkLightness{0.9}
\SetWatermarkFontSize{5cm}
\SetWatermarkScale{0.5}
\SetWatermarkText{Seminários}



%-----------------------------------------------------------------------------%
%	Informações sobre o PDF
%-----------------------------------------------------------------------------%

\pdfinfo{%
  /Title    (GEO232 - Seminários)
  /Author   (Ju Leonel)
  /Creator  (Ju Leonel)
  /Producer (Ju Leonel)
  /Subject  (Intro oceanografias)
  /Keywords (Intro oceanografia, Seminários)}

%-----------------------------------------------------------------------------%
%	Documento
%-----------------------------------------------------------------------------%
\title{GEO232 - Introdução à Oceanografia - IGEO-UFBA}
\author{\vspace{-10ex}}
\date{\vspace{-10ex}}

\begin{document}

  \maketitle
  %\doublespacing
  \onehalfspace

  \begin{tabular*} {0.9\textwidth}{@{\extracolsep{\fill} } l l}
    \hline
    Professora: Juliana Leonel & Atendimento: Sextas-feiras \\
    E-mail: \href{mailto:jleonel@ufba.br}{jleonel@ufba.br} & Horário atendimento: 13:00-14:00 \\
    Aulas: Terças e Quintas-Feiras & Local atendimento: IGEO - Sala 10 - 2\textsuperscript{o} andar\\
    Horário- Aulas: 10:40 - 12:30 & Homepage: \url{http://juoceano.github.io/geochemistry}\\
    \hline
  \end{tabular*}

  \vspace{3ex}

  \centerline{ \textbf{Seminários}}

  Os seminários deverão ser feitos em duplas. Cada dupla deverá passar o tema do seminários e suas respectivas identificações até o dia 19 de Março.


  \section* {1. Temas}
    \noindent
    Os temas dos seminários devem englobar assuntos que não teremos tempo de discutir amplamente em sala de aula. Algumas sugestões:

   a) erosão costeira;

   b) sobrepesca;

   c) uso de satélites na oceanografia;

   d) exploração de recursos minerais marinhos;

   e) construção de {\it Resorts} e Hotéis próximos de zonas de alta sensibilidade ambiental;

   f) balneabilidade de praias em zonas urbanas;
   
   g) 
   
  \section* {2. Apresentações}
    \noindent

    a) apresentações: 15 minutos (os alunos que ultrapassarem o tempo serão interrompidos e avaliados pela parte que apresentaram);

    b) perguntas/discussões;

    c) todos as duplas deverão fazer ao menos uma pergunta durante a pelo menos uma das apresentações;

    d) para as apresentações podem ser usados slides no datashow, quadro negro, cartazes etc. Enfim, a dupla pode usar os recursos que achar necessário para sua apresentação.


   \section* {3. Avaliação}
   \noindent

    Abaixo estão listados alguns dos critérios de avaliação do trabalho.

    \textbf{1. Estrutura:} há lógica e progressão de ideias na apresentação?

    \textbf{2. Clareza:} as sentenças são concisas? O vocabulário usado esta correto? Um expectador comum compreende as ideias apresentadas?

     \textbf{3. Relevância das informações:} todas as informações presentes tem relevância para o trabalho ou algumas (ou muitas) foram usadas apenas para aumentar o tamanho da apresentação?

    \textbf{4. Entendimento sobre o assunto:} os alunos demonstraram conhecimento robusto sobre o assunto? Os alunos não se ativeram apenas aos conceitos básicos, mas trouxeram  novos conhecimentos na apresentação?

    \textbf {5. Revisão bibliográfica:} foi realizada uma revisão profunda sobre o assunto? As informações apresentadas são atuais?


\vspace{10ex}

%\clearpage %termina o texto e tudo que estiver flutuante se encaixa ali
%\newpage % começa uma nova página
%\bibliographystyle{chicago} % estilo que vai sair a bibliografia (posso usar outros)
%\bibliography{references_ju} %entre chaves vai o nome do arquivo de referencias

\end{document}
