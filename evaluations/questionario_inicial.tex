\RequirePackage{fix-cm}  % Fix Font shape `OT1/cmr/m/n' size substitution.
%-----------------------------------------------------------------------------%
%	Packages & Other Configurations
%-----------------------------------------------------------------------------%
\documentclass[a4paper,10pt]{article}
\usepackage[top=0.5in, bottom=0.5in, left=0.9in, right=0.3in]{geometry}

\usepackage[utf8]{inputenc} %add acents
\usepackage{setspace} % command \doublespacing etc...
\usepackage{lineno} % number lines
\usepackage{epsf,epsfig} % includegraphics [pdf, png etc]
\usepackage{amsmath} %adicionei esse pacote pra vc poder usar o draft%
\usepackage{textcomp} %símbolos de texto
\usepackage{natbib} % bibtex - adicionar referencia
% \usepackage{url} % for bibtex - configuracoes de urls
\usepackage{tabularx} % for tables
\usepackage[hidelinks]{hyperref}  % Add URL links.
% \usepackage[bookmarks=false,colorlinks=true,urlcolor={green},linkcolor={green},pdfstartview={XYZ null null 1.22}]{hyperref} %all references
\usepackage{multicol}


%-----------------------------------------------------------------------------%
%	Adicionar a Watermark
%-----------------------------------------------------------------------------%
\usepackage{draftwatermark}
\SetWatermarkAngle{45}
\SetWatermarkLightness{0.9}
\SetWatermarkFontSize{5cm}
\SetWatermarkScale{0.5}
\SetWatermarkText{Questionário Inicial}



%-----------------------------------------------------------------------------%
%	Informações sobre o PDF
%-----------------------------------------------------------------------------%

\pdfinfo{%
  /Title    (GEO232 - Questionário Inicial)
  /Author   (Ju Leonel)
  /Creator  (Ju Leonel)
  /Producer (Ju Leonel)
  /Subject  (Questionário Inicial)
  /Keywords (Questionário Inicial, intro oceanografia)}

%-----------------------------------------------------------------------------%
%	Documento
%-----------------------------------------------------------------------------%
\title{GEO232 Introdução à Oceanografia - Questionário Inicial}
\author{\vspace{-10ex}}
\date{\vspace{-10ex}}

\begin{document}

  \maketitle
  %\doublespacing
  \onehalfspace

%   \begin{tabular*} {0.9\textwidth}{@{\extracolsep{\fill} } l l}
%     \hline
%     Professora: Juliana Leonel & Atendimento: Sextas-feiras \\
%     E-mail: \href{mailto:jleonel@ufba.br}{jleonel@ufba.br} & Horário atendimento: 13:00-14:00 \\
%     Aulas: Quintas e sextas-feiras & Local atendimento: IGEO - Sala 10 - 2\textsuperscript{o} andar\\
%     Horário- Aulas: 7:00 - 10:40 & Homepage: \url{http://juoceano.github.io/geochemistry}\\
%     \hline
%   \end{tabular*}
%
%   \vspace{3ex}

Para cada um dos termos abaixo classifique seu nível de conhecimento/familiarização de acordo com a legenda abaixo:

\begin{multicols}{2}

I. nunca ouvi falar.

II. ouvi falar, mas não lembro o que é.

III. tenho uma ideia do que é.

IV. sei o que é e posso explicar o que é.

\end{multicols}

\begin{multicols}{2}

1. crosta terrestre $\rule{1cm}{0.15mm}$

2. manto $\rule{1cm}{0.15mm}$

3. litosfera $\rule{1cm}{0.15mm}$

4. astenosfera $\rule{1cm}{0.15mm}$

5. crosta oceânica $\rule{1cm}{0.15mm}$

6. densidade $\rule{1cm}{0.15mm}$

7. mesosfera $\rule{1cm}{0.15mm}$

8. deriva continental $\rule{1cm}{0.15mm}$

9. placas tectônicas $\rule{1cm}{0.15mm}$

10. espalhamento do fundo oceânico $\rule{1cm}{0.15mm}$

11. margem continental $\rule{1cm}{0.15mm}$

12. plataforma continental $\rule{1cm}{0.15mm}$

13. talude $\rule{1cm}{0.15mm}$

14. isostasia $\rule{1cm}{0.15mm}$

15. cordilheira meso-oceânica $\rule{1cm}{0.15mm}$

16. tectonismo $\rule{1cm}{0.15mm}$

17. sísmica $\rule{1cm}{0.15mm}$

18. Ciclo de Wilson $\rule{1cm}{0.15mm}$

19. zona de subducção $\rule{1cm}{0.15mm}$

20. Rift Valley $\rule{1cm}{0.15mm}$

21. placas convergentes $\rule{1cm}{0.15mm}$

22. placas divergentes $\rule{1cm}{0.15mm}$

23. falhas transformantes $\rule{1cm}{0.15mm}$

24. sedimentos terrígenos $\rule{1cm}{0.15mm}$

25. sedimentos biogênicos $\rule{1cm}{0.15mm}$

26. sedimentos vulcanogênicos $\rule{1cm}{0.15mm}$

27. sedimentos cosmogênicos $\rule{1cm}{0.15mm}$

28. nódulos polimetálicos $\rule{1cm}{0.15mm}$

29. correntes turbidíticas $\rule{1cm}{0.15mm}$

30. átomo $\rule{1cm}{0.15mm}$

31. íon $\rule{1cm}{0.15mm}$

32. molécula $\rule{1cm}{0.15mm}$

33. calor específico $\rule{1cm}{0.15mm}$

34. salinidade $\rule{1cm}{0.15mm}$

35. intemperismo $\rule{1cm}{0.15mm}$

36. propriedades conservativas $\rule{1cm}{0.15mm}$

37. vapor de pressão $\rule{1cm}{0.15mm}$

38. haloclina $\rule{1cm}{0.15mm}$

39. picnoclina $\rule{1cm}{0.15mm}$

40. termoclina $\rule{1cm}{0.15mm}$

41. pH $\rule{1cm}{0.15mm}$

42. pressão atmosférica $\rule{1cm}{0.15mm}$

43. efeitos de Coriolis $\rule{1cm}{0.15mm}$

44. ressurgência marinha $\rule{1cm}{0.15mm}$

45. circulação termohalina $\rule{1cm}{0.15mm}$

46. massas d'água $\rule{1cm}{0.15mm}$

47. diagrama TS $\rule{1cm}{0.15mm}$

48. maré de sizígia $\rule{1cm}{0.15mm}$

49. maré de quadratura $\rule{1cm}{0.15mm}$

50. maré enchente $\rule{1cm}{0.15mm}$

51. maré vazante $\rule{1cm}{0.15mm}$

52. altura de onda $\rule{1cm}{0.15mm}$

53. comprimento de onda $\rule{1cm}{0.15mm}$

54. período de onda $\rule{1cm}{0.15mm}$

55. onda interna $\rule{1cm}{0.15mm}$

56. maré semidiurna $\rule{1cm}{0.15mm}$

57. organismos bentônicos $\rule{1cm}{0.15mm}$

58. organismos pelágicos $\rule{1cm}{0.15mm}$

59. organismos planctônicos $\rule{1cm}{0.15mm}$

60. zona abissal $\rule{1cm}{0.15mm}$

61. zona nerítica $\rule{1cm}{0.15mm}$

62. zona fótica $\rule{1cm}{0.15mm}$

63. zona inter-maré $\rule{1cm}{0.15mm}$

64. quimiossíntese $\rule{1cm}{0.15mm}$

65. alça microbiana $\rule{1cm}{0.15mm}$

66. diatomáceas $\rule{1cm}{0.15mm}$

67. produção primária $\rule{1cm}{0.15mm}$

68. produtividade secundária $\rule{1cm}{0.15mm}$

69. migração diária {\it dial} $\rule{1cm}{0.15mm}$

70. organismos filtradores $\rule{1cm}{0.15mm}$

71. estuários $\rule{1cm}{0.15mm}$

72. manguezal $\rule{1cm}{0.15mm}$

73. marisma $\rule{1cm}{0.15mm}$

74. fontes hidrotermais $\rule{1cm}{0.15mm}$

75. recife de corais $\rule{1cm}{0.15mm}$

76. zona de exclusividade econômica $\rule{1cm}{0.15mm}$

77. hidratos de metano $\rule{1cm}{0.15mm}$

78. profundidade de compensação do carbonato $\rule{1cm}{0.15mm}$


\end{multicols}






%\clearpage %termina o texto e tudo que estiver flutuante se encaixa ali
%\newpage % começa uma nova página
%\bibliographystyle{chicago} % estilo que vai sair a bibliografia (posso usar outros)
%\bibliography{references_ju} %entre chaves vai o nome do arquivo de referencias

\end{document}
